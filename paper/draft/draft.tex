\documentclass[11pt,a4paper]{article}
\usepackage[margin=1in]{geometry}
\usepackage[utf8]{inputenc}
\usepackage{booktabs} % for toprule, midrule and bottomrule
\usepackage{adjustbox}
\usepackage{amsmath}
\usepackage{etoolbox}
\usepackage{setspace} % for \onehalfspacing and \singlespacing macros
\usepackage[hidelinks]{hyperref}
\usepackage{array}
\usepackage{graphicx}
\usepackage{setspace}
\usepackage{caption}
\usepackage{pdflscape}
\usepackage[labelfont={bf}]{caption}
\usepackage{tabularx}

% set space
\doublespacing

% booktabs
\setlength\heavyrulewidth{0.2ex}

% images
\graphicspath{ {D:/Users/saketh/Documents/HonorsThesisData/Exhibits/} }

% array
\newcolumntype{L}[1]{>{\raggedright\let\newline\\\arraybackslash\hspace{0pt}}m{#1}}
\newcolumntype{C}[1]{>{\centering\let\newline\\\arraybackslash\hspace{0pt}}m{#1}}
\newcolumntype{R}[1]{>{\raggedleft\let\newline\\\arraybackslash\hspace{0pt}}m{#1}}

% row fonts


% hyperlinks
\hypersetup{
	colorlinks=false
}

% bibliography
\makeatletter
\renewenvironment{thebibliography}[1]
{\section*{References}%
	\@mkboth{\MakeUppercase\refname}{\MakeUppercase\refname}%
	\list{}%
	{\setlength{\labelwidth}{0pt}%
		\setlength{\labelsep}{0pt}%
		\setlength{\leftmargin}{\parindent}%
		\setlength{\itemindent}{-\parindent}%
		\@openbib@code
		\usecounter{enumiv}}%
	\sloppy
	\clubpenalty4000
	\@clubpenalty \clubpenalty
	\widowpenalty4000%
	\sfcode`\.\@m}
{\def\@noitemerr
	{\@latex@warning{Empty `thebibliography' environment}}%
	\endlist}
\makeatother

% etoolbox
\AtBeginEnvironment{quote}{\singlespacing\small}

\begin{document}
	
\title{The Effect of Payment for Order Flow on Order Routing}

\author{Saketh Aleti}

\maketitle


\section{Introduction}	

One of the most significant changes in US equity markets in the last decade is the emergence of widespread competition for retail order flow\footnote{ Retail order flow refers to orders for securities that originate from retail investors as opposed to institutional investors such as banks and hedge funds}. About 20 years ago, stocks listed on an exchange such as NASDAQ were only traded on that exchange. Today, when a broker receives an order, they can redirect the order to one of hundreds of locations or internalize\footnote{ Internalization is the process by which a broker executes an order against their own inventory, so they can profit from the difference between the order's price and the market price} it. If they choose to redirect the order, it may go to an exchange, a market maker, or an electronic communications network. 
The SEC refers to these venues as \textit{market centers}; they define a market center as ``any exchange market maker, OTC market maker, alternative trading system, national securities exchange, or national securities association" as per the Securities Exchange Act of 1934.
Most venues offer brokers some rebate, usually less than a penny per share, in order to attract order flow.  The process by which market centers offer rebates for orders from broker-dealers is called \textit{payment for order flow}. This practice remains controversial, since it allows brokers to effectively sell their client's orders to the highest bidder rather than sending them to the market center with the best execution. At the same time, some have argued that payment for order flow has come with benefits such as greater exchange competition (Angel, Harris, and Spatt, 2011). 


According to an industry notice from the National Association of Securities Dealers, now the self-regulatory agency FINRA,


\begin{quote}
	The traditional non-price factors affecting the cost or efficiency of executions should also continue to be considered; however, broker-dealers must not allow an order routing inducement, such as payment for order flow or the opportunity to trade with that order as principal, to interfere with its duty of best execution. (2001)
\end{quote}
The SEC has made similar statements and codified certain requirements for the processing of order flow. Between 2005 and 2007, the SEC updated the Securities Exchange Act of 1934 with Regulation National Market System (Reg NMS); this regulation introduced new rules for order protection, intermarket access, sub-penny pricing, and market data. In particular, according to the SEC's Order Protection Rule of Reg NMS, orders must be executed at the National Best Bid or Offer (NBBO) or better; the best bid is the highest price a buyer will accept and the best offer is the lowest price a seller will accept. This rule establishes price priority, so brokers cannot\footnote{ There are exceptions for this rule which are aimed at High Frequency Traders, but this is irrelevant for retail order flow.} execute their client's orders at a worse price than the NBBO. However, when multiple exchanges are trading a stock at the NBBO, the broker can choose which venue to route to. 

In such a case, brokers may decide where to route based on the rebate they receive for directing order flow. This involves responding to one of two fee structures: either payment for order flow or maker-taker fees. The latter simply involves paying a fee for removing liquidity (sell orders) and receiving a rebate for adding liquidity (buy orders). This fee/payment structure is particularly targeted towards limit orders. On the other hand, payment for order flow consists of purchasing orders independent of whether they are buy or sell; generally, this is targeted towards all order types. 
For the sake of a retail investor, it would be best if brokers routed their clients' orders to maximize their execution quality; 
for instance, they ought to optimize execution quality factors such as price improvement, execution speed, execution probability, and so on. Brokers are incentivized to do so, because offering a better service should win them more clients; also, legally, they are supposed to route orders to where they would receive the best execution. But, there exists enough leniency within the SEC's definition of a "best execution" under Reg NMS for brokers to route according to their own objectives and offer a post hoc justification. 

My thesis aims to identify how the presence of payment for order flow affects order routing. 
So, I examine the responsiveness of different subgroups of brokers to changes in the execution quality of market centers. 
Some brokers happen to accept payment for order flow while others do not; this makes it possible to compare the order routing of \textit{paid} (payment-accepting) brokers versus \textit{unpaid} brokers, and thus the identify the effect of payment for order flow. 
I expect that all brokers route a greater percentage of orders towards market centers with better factors of execution quality such as higher price improvement or faster execution speed. In particular
In particular, I hypothesize that paid brokers are less responsive than unpaid brokers to changes in execution quality. That is, payment for order flow has a negative effect the executions of retail orders directed by paid brokers. This research expands upon previous research in market microstructure that studies the effects of rebates and fees offered firms looking to capture order flow.  


\pagebreak	
\section{Literature Review}	

\subsection{The Theoretical Foundation for Order Rebates}

Early research on the effects of payment for order flow is still relatively recent given that the practice emerged in the 1980s. At that time, order flow payments were as high as a penny per share, since major exchanges would trade in $12.5$ cent increments. As a result, the early literature, such as Chordia and Subrahmanyam (1995) and Kandel and Marx (1999), argued that payment for order flow was caused by higher tick sizes. However, as tick sizes got smaller, payments did not disappear but simply diminished in size. Now, newer papers like Parlour and Rajan (2001) and Battalio and Holden (2001) present theoretical models for the existence of payments for order flow persisting in continuous price settings. 

In particular, Battalio and Holden find that brokers can maximize profits by using their internal data on their clients to identify uninformed order flow; if a broker predicts that their client is uninformed, they internalize the order and profit from the spread. On the other hand, their model assumes that third parties, such as market centers, purchase order flow based on "externally-verifiable" characteristics; this is a realistic assumption given that third parties do not have access to client data before they purchase an order. These factors motivate third parties to pay for order flow, since they will only have access to potentially uninformed order flow if they can compete with internalization. The former paper, Parlour and Rajan (2001), develops another theoretical model consisting of an infinite horizon game of liquidity-provision. This game consists of competing market makers, brokers, and investors; market makers supply liquidity, investors demand or supply liquidity, and brokers act as intermediaries. Their model finds that scenarios where payment for order flow is allowed results in lower consumer and social welfare than the counterfactual. Additionally, its existence results in a welfare transfer from liquidity takers to suppliers, which they define as those submitting market orders versus those sending limit orders. Both models hold the number of market centers fixed, while there also exists literature on the effects of greater competition as well. 

\subsection{The Effects of Order Rebates}

Angel, Harris, and Spatt (2011) note that Regulation NMS has resulted in significantly greater competition among market centers. However, Dennert (1993) and Duta and Madhavan (1997) find that competition between market centers can actually have mixed effects for retail investors. Firstly, Dennert (1993) models price competition where prices are quoted by competiting dealers facing some unknown future demand. Then, introducing informed and uninformed traders to the model, Dennert finds that increasing the number of market makers increases their individual risk exposure. As a result, bid-ask spreads increase and liquidity traders face higher transaction costs. So, there exist cases where liquidity traders, who are generally uninformed, would be better off in monopolistic markets than competitive ones. Next, Duta and Madhavan (1997) present a model where multiple market makers compete with one another resulting in higher spreads; in some ways, this effect is similar to what would occur if the market makers colluded and fixed prices. They suggest that this outcome is due to current institutional arrangements, and investors would be better off if price priority was eliminated, internalization was discouraged, and order flow payments were eliminated. 

Other papers find similar results regarding the effect of order rebates. Cimon (2016) presents a theoretical model of limit order trading and finds that significant spreads in the fee/rebate for maker-taker venues cause brokers to reroute orders to venues with poor execution quality. This model consists of two periods where brokers decide where to route their clients marketable orders. The paper then provides a theoretical basis for two empirical results: exchanges with higher rebates will endogenously have worse fill rates and, when trading fees are sufficiently different, brokers will prioritize rebates over their clients' orders fill rates. Maglaras, Moallemi, and Zheng (2015) develop a different model where, instead of brokers acting as an intermediary, investors route their own orders and can pay fees or accept rebates for their order flow. The model finds that limit orders sent to high fee venues still face worse execution quality even when investors themselves are making the decision. Additionally, when fill rates between high and low fee venues are significantly close, routing to capture rebates is welfare improving for investors. In other cases, the rebates offered by high fee venues are not enough to offset their subpar execution quality. In short, both papers imply that maximizing rebate revenue is not equivalent to maximizing execution quality. 

Lastly, there exists literature examining the effect of maker-taker and payment for order flow models on execution quality. To begin, we have Battalio, Shkilko, and Van Ness (2016), which examines differences between venues that pay for order flow and those that use the maker-taker model in US options markets. Using trade and quote data for options from OPRA, they find that the cost of liquidity is approximately 40 basis points lower in markets that pay for order flow than those that use maker-taker. Additionally, they find that high priced options face lower spreads in maker-taker exchanges; therefore, they argue that brokers cannot fulfill their best execution obligation if they route all their orders to venues that use the payment for order flow model, which would maximize broker revenue. Analyzing US equity markets instead, Battalio, Corwin, and Jennings (2016), find empirical evidence that brokers who route to maximize rebates cannot be maximizing execution quality. Using a proprietary data set of orders, they develop a model which they then further investigate with the larger TAQ dataset. Battalio, Corwin, and Jennings find that four major retail brokers "appear to route all nonmarketable limit orders to the venue offering the largest rebate." Additionally, based on results from their proprietary data set, they find that limit order execution quality is negatively correlated with an exchange's rebate. Generalizing to the TAQ dataset, they also find that order rebates have greater impacts on low-priced stocks. In any case, these papers develop an empirical foundation for the argument that maximizing rebate revenue does not maximize execution quality. 

Based on previous literature, we expect order routing that prioritizes rebate revenue to be subpar in execution quality. As stated above, this has both a theoretical foundation and an empirical foundation. This paper aims to estimate the effect of payment for order flow on the market share of market centers. Given that the latter is measured by order flow itself, this question can also be rephrased as finding the elasticity of the rebate price with respect to the quantity of orders. So, while the literature explains the effects of revenue-maximizing order routing, this research will estimate the extent of order routing that is actually done to maximize rebates. Thus, the result will further develop the empirical literature and may offer new questions for theory. 






	
\pagebreak	
\section{Data Description}

	For my methodology, I use data from two sources: 605 disclosures and 606 disclosures. After collecting the data, I merge both into a dataset where each observation belongs corresponds to a specific broker, market center, exchange, order type, and time period. Because the data sets are in different time frequencies, I consolidate each to quarterly before merging. 
	
	\subsection{605 Data}
	
		Firstly, I obtain the 605 disclosures from each market center's webpage. The SEC requires market centers to disclose information every month regarding the quality of executions for each stock; these disclosures are referred to as 605 filings in accordance with Rule 605 of Reg NMS. The 605 data contains the volume of each stock for each order type that passes through the market center. Additionally, it breaks down this volume into orders that received price improvement, orders that were at-the-quote, and orders that were outside-the-quote. When an order receives price improvement, it means that the buyer or seller received a better price than what they were willing to pay or what the market offered (the NBBO); for instance, if the NBBO for \$AAPL is at $100/101$, market centers that execute trades between these prices provide price improvement. This data is further broken down into the number of price improved shares for each stock, the average amount of price improvement, and the average time a price improved order took to execute. Similarly, market centers report the average time and shares of ATQ and OTQ orders as well; ATQ refers to orders at the NBBO while OTQ refers to orders outside the NBBO. This data is necessary in constructing my independent variables: measures of execution quality. Also, since 605 data is reported on a monthly and per stock basis, I convert it to a quarterly and per exchange basis to match the 606 data. Lastly, I use execution quality statistics for orders with the size code 21, which refers to orders between 100-499 shares; this is the smallest order size category available and likely best represents executions for the orders of retail investors. 
		
	\subsection{606 Data}
	
		Similarly, the SEC requires broker-dealers to disclose where they route their orders and the average payment that they receive; these disclosures are referred to as 606 filings. The 606 data shows the percent of orders, by order type, that was routed to each market center. For example, the following is a table from TD Ameritrade's 606 disclosure for NASDAQ stocks during the third quarter of 2017. 
		

		\vspace{1em}

		\begin{table}[h]
			
			\caption{Cross-section 605 Data (TD Ameritrade, NASDAQ, 2017Q3) }
			\centering
			\footnotesize

				\begin{tabular}{@{} l   C{2.5cm}  C{2.5cm}  C{2.5cm}  C{2.5cm} @{}}
					\toprule
					& \multicolumn{4}{c}{{\textbf{Order Type}}} \\ 
					\cline{2-5} \\ [-1.5ex]
					\textbf{Routing Venue} & Total & Market  & Limit &  Other  \\ \midrule
					TD Ameritrade Clearing & 77\%  & 85\% &  74\%  & 72\%   \\ 
					Citadel Execution Services &  13\% &  9\%  & 14\%  & 17\%   \\
					Virtu Americas, LLC  & 6\% &  4\%  & 6\%  & 8\%   \\ 
					Citi Global Markets  & 3\%  & 0\%  & 5\% &  0\%   \\ 
					Two Sigma Securities, LLC  & 1\%  & 2\%  & 1\%  & 3\%   \\ 
					\bottomrule
				\end{tabular}

		
		\end{table}
		
		Also, note that TD Ameritrade redirects a majority of its orders to its own clearing corporation, which itself produces a 606 disclosure documenting its order flow; this is a fairly common practice among the larger brokers. In order to account for order flow from a broker to its subsidiary clearing corp, I take the order flow payments made to the clearing corporation (found in their seperate 606 filing), weight them by the share of orders sent to the clearing corp, and add them back to the order flow payments to the broker. This ensures that order flow sent to subsidiaries is accounted for in the final dataset. For instance, if TD Ameritrade Clearing sends 50\% of its orders to Citadel, then the total percent of orders routed to Citadel is 51.5\% (77\% * 50\% + 13\%).  
		
		Along with this information, the disclosure states the average payment received by TD Ameritrade from each routing venue; this is the payment for order flow. However, this variable is a deterministic function of the order routing already present in the data. That is, market centers pay different amounts for different types of orders. The average payment reported by brokers is the total payment received divided by the number of shares routed to a certain market center. So, if a market center offers higher rebates for market orders than limit orders and a broker routes more market orders their way in some future time period, then the average payments reported would increase even though the actual rebates offered may not have changed. This makes it impossible to draw any information from the average payments statistic without an endogeneity problem. Additionally, more specific data regarding payment for order flow is unavailable, so there is no way of extracting any information from average payments alone. Instead, I simply keep track of which brokers offer order rebates with a dummy variable; brokers, legally, must disclose if they are receiving payment for order flow, so this data is available. I use this later on for a subgroup analysis. 
		
		It is also important to note that order routing requires particular infrastructure to process which limits the universe of market centers that brokers can route; this requires adjusting the raw 606 data. Returning to the TD Ameritrade example, they route to Citadel, Virtu, Citi, and Two Sigma, although many other potential market centers exist. One approach to filling in the discluded entries would be to populate the 606 data with 0\% entries for all market centers available from the 605 data. However, this would ignore the issue of infrastructure. A broker may set up a connection to a new market center if they believed that they would receive marginally better execution quality by doing so. And, setting up a connection is a fixed cost and the connection likely persists for some time even after brokers cease to route to a market center. So, I adjust the 606 data to include missing zeros by only adding entries for market centers that a broker has routed to in the past two quarters. This ensures that regressions on this data avoid selection bias while taking infrastructure into consideration. 
		
		Lastly, some brokers did not accept payment for order flow but were still included under the subgroup \textit{POF brokers}. This was done when brokers routed to clearing corporations that received payment for order flow. For instance, Hollencrest Securities receives no POF, but contracts Charles Schwab to route its orders; the latter do accept POF. Hence, Hollencrest's order routing is still implicitly under the influence of rebates although it receives none. So, I use the term \textit{POF} or \textit{paid} in the sense that these types of brokers' order routing is affected by POF whether or not they actually receive any direct financial compensation.
		
		In total, the 606 data provides order routing data, or "market share" within brokers, for each market center specified by exchange and order type on a monthly basis. 


	
\pagebreak
\section{Methodology}

	\subsection{Model Formulation}

	Suppose that brokers route orders based on a the market center's execution quality and its payment for order flow; realistically, these are the only possible variables a profit-maximizing firm would take into account besides for some sort of personal relationship with a market center. That sort of effect would be fixed over time, so it can be modeled with a constant term. So, in total, we have fixed effects represented by $\alpha$, a vector of execution quality factors represented by $X$, and the level of payment for order flow denoted by $Z$; these variables determine $Y$ which denotes market share - the percent of orders routed by a broker to a particular market center. 
	\begin{equation}
	Y_{i, j, k, l, t} = \alpha_{i,j,k,l} +  X_{j, k, l, t} \cdot \beta +   Z_{j, l, t} \cdot \gamma + \epsilon
	\end{equation}	
	
	The subscript $i$ denotes the broker, $j$ denotes the market center, $k$ denotes the exchange, and $l$ denotes the order type. Lastly, we have $\epsilon$ representing the error term.
	

	In this model, the term $Z$, payment for order flow, is unknown because there is no data available to construct it. This could cause omitted variable bias if $\gamma \neq 0$ and $cov(X, Z) \neq 0$. However, although the former I hypothesize to be true, I believe that the latter is not. This is because $Z_{j, l, t}$ states the level of rebate offered for order flow at some point in time $t$. For it to have a causal relationship with $X_{j, k, l, t}$, we would need market centers either reacting to execution quality shifts by changing rebates and/or responding to rebate revenue with changes in execution quality. The latter is impossible, because market centers cannot simply choose how well trades are executed on their markets. Better trade-matching algorithms may lead to better executions, but market centers cannot "undo" investments into improving their execution quality nor are these straightforward investments with consistent, short-term returns in execution quality. So, if rebate revenue increased, a market center cannot cut costs by reducing execution quality; and, if it decreased, it is improbable that a market center could simply better its execution quality in the same period. Additionally, improvements in execution quality tend to be a result of technological process (eg: faster computers); its non-trivial for a market center to improve execution quality by any premeditated amount. On the other hand, it is also unlikely that market centers can respond to execution quality changes by changing their rebates. This is because market centers cannot change their rebates very often; there's a significant amount of paperwork involved and it can take several quarters for changes in the rebate structure to actually go through. So, market centers cannot react well to changes in execution quality by changing the prices they offer for rebate flow. Moreover, it is very difficult to predict when execution quality will fall, and, even if predicted, it may be a result of systemic problems in the market rather than being particular issue with the market center itself. In short, $cov(Z, X)$ is likely $0$. Moreover, $\gamma$ will trivially equal $0$ for unpaid brokers, so regressions on their subgroup will remain consistent even without this assumption. 
	
	Second, with respect to the factors of execution quality, $X$, I construct this variable using data on price improvement and execution time through the 605 disclosures. Considering price improvement, I have data on the number of price-improved shares, the average amount of price improvement, and the average time that price-improved shares took to execute. I use the latter two in my regressions, and I use the former to construct the percent of shares that were price-improved. Additionally, I include an interaction term between percent of price-improved shares and average price improvement; their product is the expected amount of price improvement. Additionally, since the data shows the average time that all types of executions took, I construct a variable for the average time that a share takes to execute at a market center. These variables are denoted in my regression results as $PrImp\_Pct$, $PrImp\_AvgAmt$, $PrImp\_ExpAmt$, $PrImp\_AvgT$, and $All\_AvgT$. I do not include both $PrImp\_AvgT$ and $All\_AvgT$ in the same regression, since they are highly collinear. Additionally, I consider average effective spreads, the average difference between the NBBO spread (ask - bid) and the trade price, in my regression; tighter spreads generally indicate better executions. In total, I run several regressions using some combination of terms for the level of price improvement, a term for average time taken, and a term for spreads; I use the variables with the best explanatory power to construct the $X$ variable. 
	
	\begin{equation}
	Y_{i, j, k, l, t} \,=\, \alpha_{i,j,k,l} \,+\,  X_{j, k, l, t} \cdot \beta \,+\,  (D_i \cdot X_{j, k, l, t}) \cdot \gamma\, +  u
	\end{equation}	
	\begin{equation}
	Y_{i, j, k, l, t} = \alpha_{i,j,k,l} +  X_{j, k, l, t} \cdot \beta +  v
	\end{equation}	

	Lastly, since $Y$  is defined as market share, this model is censored at 0 and 1. In the data, there are no cases where a broker routes 100\% of its orders to any market center, so only censoring at 0\% practically applies. In this case, a Tobit model would be appropriate. However, some argue that the assumption of normality is more problematic than the inconsistency caused by running OLS on censored data. So, I run OLS, Tobit, and a semiparametric model on the data, and compare the results. With respect to the OLS regression, the unobservable fixed effects may be correlated with $X$ and the large quantity of constant terms may create inconsistency. So, I consider both random effects and fixed effects and choose the appropriate model using a Hausman test. In these models, I use function forms resembling equation (2) where I regress on $X$ and $D_i \cdot X$ where $D_i$ is an indicator variable for whether a firm accepts payment for order flow. The significance of $\gamma \neq 0$ then determines whether the coefficients for paid brokers are significantly different than unpaid brokers. And, lastly, with respect to the semiparametric model, I regress using equation (3) and a Gaussian kernel to find $\theta = \beta / \beta_1$. Since this regression returns only the ratios between coefficients, I treat coefficient $\beta_1$ on the amount of price improvement, a element of execution quality, as a given variable in all models. Then, I find the standard errors of the other variables. I perform the semiparametric regression on paid and unpaid brokers seperately and the compare the significance of the coefficients in each and differences in marginal effects. 
	
	


\pagebreak	
\section{Results}

\vspace{5em}
	
\begin{table}[!htbp] 
	\captionsetup{font=normal}
	\caption{OLS Regression Results for Market Orders} 
	\label{} 
	\centering
	\small
	\begin{tabular}{@{\extracolsep{1em}}lcccc} 
		\\[-1.8ex]
		\toprule \\[-1.8ex] 
		& \multicolumn{4}{c}{\textbf{Market Share}} \\ 
		\cline{2-5} 
		\\[-1.8ex] & (1) & (2) & (3) & (4)\\ 
		\hline \\[-1.8ex] 
		PrImp\_Pct & -0.0959 &  & -0.0956 & \\
		& (0.0955) &  & (0.0955) & \\ [1.5ex]
		PrImp\_Pct$*$Rebate\_Dummy & -0.3085$^{*}$ &  & -0.3079$^{*}$ & \\
		& (0.1399) &  & (0.14) & \\ [1.5ex]
		PrImp\_AvgAmt & 8.5362$^{***}$ &  & 8.4626$^{***}$ & \\
		& (2.0766) &  & (2.0863) & \\ [1.5ex]
		PrImp\_AvgAmt$*$Rebate\_Dummy & -4.415 &  & -4.8303 & \\
		& (3.0109) &  & (3.0307) & \\ [1.5ex]
		PrImp\_ExpAmt &  & 8.0954$^{***}$ &  & 8.0103$^{***}$\\
		&  & (2.2921) &  & (2.2992)\\ [1.5ex]
		PrImp\_ExpAmt$*$Rebate\_Dummy &  & -8.5685$^{**}$ &  & -8.9484$^{**}$\\
		&  & (3.1505) &  & (3.165)\\ [1.5ex]
		PrImp\_AvgT & -0.0196$^{**}$ & -0.0195$^{**}$ &  & \\
		& (0.0062) & (0.0064) &  & \\ [1.5ex]
		PrImp\_AvgT$*$Rebate\_Dummy & 0.0112 & 0.0147 &  & \\
		& (0.0079) & (0.008) &  & \\ [1.5ex]
		All\_AvgT &  &  & -0.0084$^{**}$ & -0.0084$^{**}$\\
		&  &  & (0.003) & (0.0031)\\ [1.5ex]
		All\_AvgT$*$Rebate\_Dummy &  &  & 0.0061 & 0.0078\\
		&  &  & (0.0041) & (0.0041)\\ [1.5ex]
		\hline \\[-1.8ex] 
		Model & FE & FE & FE & FE \\ 
		N & 2982 & 2982 & 2982 & 2982 \\ 
		R$^{2}$ & 0.027 & 0.007 & 0.026 & 0.006 \\ 
		Adjusted R$^{2}$ & 0.025 & 0.006 & 0.024 & 0.005 \\ 
		F Statistic & 6.314$^{***}$ & 4.609$^{**}$ & 5.787$^{***}$ & 4.082$^{**}$ \\ 
		\bottomrule \\[-1.8ex] 
		\multicolumn{5}{@{}p{5.5in}}{\textit{Note: } A Haussman test was performed for each functional form between Random Effects and Fixed Effects, and the appropriate model was chosen with a cutoff of $p = 0.05$. \newline *p$\textless$0.05, **p$\textless$0.01, ***p$\textless$0.001}  \\ 
	\end{tabular} 
\end{table} 

\begin{table}[!htbp] 
	\captionsetup{font=normal}
	\caption{Tobit Regression Results for Market Orders} 
	\label{} 
	\centering
	\small
	\begin{tabular}{@{\extracolsep{1em}}lcccc} 
		\\[-1.8ex]\hline  
		\hline \\[-1.8ex]  
		& \multicolumn{4}{c}{\textit{Dependent variable:}} \\  
		\cline{2-5}  
		\\[-1.8ex] & (1) & (2) & (3) & (4)\\  
		\hline \\[-1.8ex]  
		PrImp\_Pct & $-$0.054 &  & $-$0.046 &  \\  
		& (0.102) &  & (0.102) &  \\  [1.5ex]
		PrImp\_Pct\_Rebate\_Dummy & $-$0.497$^{***}$ &  & $-$0.503$^{***}$ &  \\  
		& (0.129) &  & (0.130) &  \\  [1.5ex]
		PrImp\_AvgAmt & 14.253$^{***}$ &  & 14.180$^{***}$ &  \\  
		& (2.814) &  & (2.809) &  \\  [1.5ex]
		PrImp\_AvgAmt\_Rebate\_Dummy & $-$6.677 &  & $-$7.200 &  \\  
		& (3.815) &  & (3.806) &  \\  [1.5ex]
		PrImp\_ExpAmt &  & 14.749$^{***}$ &  & 14.623$^{***}$ \\  
		&  & (2.990) &  & (2.984) \\ [1.5ex]
		PrImp\_ExpAmt\_Rebate\_Dummy &  & $-$13.552$^{**}$ &  & $-$13.984$^{***}$ \\  
		&  & (4.210) &  & (4.200) \\  [1.5ex]
		PrImp\_AvgT & $-$0.117$^{**}$ & $-$0.116$^{**}$ &  &  \\  
		& (0.038) & (0.038) &  &  \\  [1.5ex]
		PrImp\_AvgT\_Rebate\_Dummy & 0.098$^{*}$ & 0.102$^{**}$ &  &  \\  
		& (0.038) & (0.038) &  &  \\  [1.5ex]
		All\_AvgT &  &  & $-$0.059$^{**}$ & $-$0.058$^{**}$ \\  
		&  &  & (0.019) & (0.019) \\  [1.5ex]
		All\_AvgT\_Rebate\_Dummy &  &  & 0.051$^{**}$ & 0.053$^{**}$ \\  
		&  &  & (0.020) & (0.020) \\  [1.5ex]
		Constant & $-$0.074$^{***}$ & $-$0.076$^{***}$ & $-$0.074$^{***}$ & $-$0.076$^{***}$ \\  
		& (0.006) & (0.006) & (0.006) & (0.006) \\  [0.5ex]
		\hline \\[-1.8ex]  
		Observations & 2,982 & 2,982 & 2,982 & 2,982 \\  
		Log Likelihood & $-$955.832 & $-$984.152 & $-$959.019 & $-$986.638 \\  
		Wald Test & 96.801$^{***}$ & 38.828$^{***}$ & 91.954$^{***}$ & 35.462$^{***}$ \\  
		\hline  
		\hline \\[-1.8ex]  
		\textit{Note:}  & \multicolumn{4}{r}{*p$\textless$0.05, **p$\textless$0.01, ***p$\textless$0.001} \\  
	\end{tabular} 
\end{table} 


\begin{table}[!htbp] 
	\captionsetup{font=normal}
	\caption{Semiparametric Regression Results for POF Broker Market Orders} 
	\label{} 
	\centering
	\small
	\begin{tabular}{@{\extracolsep{1em}}lcccc} 
		\\[-1.8ex]\hline  
		\hline \\[-1.8ex]  
		& \multicolumn{4}{c}{\textit{Dependent variable:}} \\  
		\cline{2-5}  
		\\[-1.8ex] & (1) & (2) & (3) & (4)\\  
		\hline \\[-1.8ex]  
		PrImp\_Pct & 1 &  & 1 &  \\  
		&  &  &  &  \\  [1.5ex]
		PrImp\_AvgAmt & 9.0835$^{***}$ &  & $-$9.1374$^{***}$ &  \\  
		& (0.00049) &  & (0.00028) &  \\  [1.5ex]
		PrImp\_ExpAmt &  &1 &  & 1 \\  
		&  &  &  &  \\ [1.5ex]
		PrImp\_AvgT & 0.0038$^{***}$ & 0.0075$^{***}$ &  &  \\  
		& (0.000002) & (0.000003) &  &  \\  [1.5ex]
		All\_AvgT &  &  & 0.0327$^{**}$ & 0.0056$^{***}$ \\  
		&  &  & (0.000013) & (0.000018) \\  [1.5ex]
		\hline \\[-1.8ex]  
		Observations & 2,982 & 2,982 & 2,982 & 2,982 \\  
		\hline  
		\hline \\[-1.8ex]  
		\textit{Note:}  & \multicolumn{4}{r}{*p$\textless$0.05, **p$\textless$0.01, ***p$\textless$0.001} \\  
	\end{tabular} 
\end{table} 

\begin{table}[!htbp] 
	\captionsetup{font=normal}
	\caption{Semiparametric Regression Results for Non-POF Broker Market Orders} 
	\label{} 
	\centering
	\small
	\begin{tabular}{@{\extracolsep{1em}}lcccc} 
		\\[-1.8ex]\hline  
		\hline \\[-1.8ex]  
		& \multicolumn{4}{c}{\textit{Dependent variable:}} \\  
		\cline{2-5}  
		\\[-1.8ex] & (1) & (2) & (3) & (4)\\  
		\hline \\[-1.8ex]  
		PrImp\_Pct & 1 &  & 1 &  \\  
		&  &  &  &  \\  [1.5ex]
		PrImp\_AvgAmt & $-$543.5313
		$^{***}$ &  & $-$125.9239$^{***}$ &  \\  
		& (2.9015) &  & (0.000072) &  \\  [1.5ex]
		PrImp\_ExpAmt &  &1 &  & 1 \\  
		&  &  &  &  \\ [1.5ex]
		PrImp\_AvgT & 0.331$^{***}$ & $<$0.00001$^{***}$ &  &  \\  
		& (0.001989) & ($<$0.000001) &  &  \\  [1.5ex]
		All\_AvgT &  &  & 0.0122$^{***}$ & $<$0.000001$^{***}$ \\  
		&  &  & (0.000009) & ($<$0.00001) \\  [1.5ex]
		\hline \\[-1.8ex]  
		Observations & 2,982 & 2,982 & 2,982 & 2,982 \\  
		\hline  
		\hline \\[-1.8ex]  
		\textit{Note:}  & \multicolumn{4}{r}{*p$\textless$0.05, **p$\textless$0.01, ***p$\textless$0.001} \\  
	\end{tabular} 
\end{table} 

%\pagebreak
%
%\phantomsection\section*{}
\pagebreak
\section*{Appendix}
\begin{center}
	\begin{table}[htbp]
		
		
		\caption*{\textbf{Table A1:} Summary Statistics for Broker Order Routing}
		\centering
		\scriptsize			
		
		\begin{tabular}{@{}lrrrrrrrrr@{}}
			%\rowfont{\normalsize}%
			\multicolumn{10}{@{} p{5.7in} @{}}{
				{\scriptsize The following table was generated using the 606 disclosures. The set of market center columns contain the average percent of orders routed to each respective market center over the available dataset. Brokers are split into two categories: POF and Non-POF. POF brokers generally, but not neccesarily, receive direct compensation for order flow routing, while Non-POF brokers do not accept payment for order flow; see the data section for a more detailed description. Additionally, this table only includes market centers whose 605 data was available for analysis. } 
			}  \\ \\ [-1.8ex]
			\toprule
			& \multicolumn{9}{c}{{\textbf{Market Center}}} \\ \\[-2.5ex] 
			\cline{2-10} \\ [-0.5ex]
			\textbf{Broker}  & BNYC & CDRG & FBCO & G1ES & SGMA & UBSS & VRTU & WOLV & Other \\
			\hline \\[-1.8ex] 
			\multicolumn{1}{@{}l}{\textit{POF Brokers}} \\ \\[-2.5ex] 
			\hline \\[-1.8ex] 
			
			Deutsche               &   5\% &   1\% &   1\% &      &   1\% &   1\% &  \textless1\% &      &     91\% \\
			Boenning Scattergood   &   2\% &  22\% &   4\% &   5\% &  10\% &  13\% &   8\% &      &     36\% \\
			Evercore Group         &      &      &  11\% &      &      &      &      &      &     89\% \\
			Credit Suisse          &  \textless1\% &  \textless1\% &  46\% &      &  \textless1\% &  \textless1\% &  \textless1\% &      &     53\% \\
			Barclays Capital       &      &  \textless1\% &  \textless1\% &      &  \textless1\% &  \textless1\% &      &      &    100\% \\
			Cambria Capital        &      &  10\% &      &      &      &  81\% &      &      &      9\% \\
			JP Morgan              &      &  12\% &   5\% &      &      &  16\% &   4\% &      &     64\% \\
			Inlet Securities       &      &  43\% &      &      &      &  27\% &      &      &     30\% \\
			BTIG                   &      &   1\% &   4\% &      &      &  \textless1\% &   1\% &      &     94\% \\
			E1 Asset Mgmt          &      &  47\% &      &      &      &  24\% &      &      &     29\% \\
			Lightspeed Trading     &      &      &  39\% &      &      &      &      &      &     61\% \\
			Two Sigma              &      &      &  \textless1\% &      &  50\% &      &      &      &     50\% \\
			Hollencrest Securities &   5\% &  15\% &      &   3\% &   7\% &   5\% &   8\% &      &     58\% \\
			Wells Fargo            &      &  \textless1\% &  \textless1\% &      &      &  \textless1\% &  \textless1\% &      &    100\% \\
			TD Ameritrade          &      &  22\% &      &   5\% &   4\% &   1\% &  11\% &   1\% &     55\% \\
			INTL FCStone           &      &  20\% &  22\% &      &  \textless1\% &  51\% &      &      &      8\% \\
			\textbf{POF Average}          &   3\% &  15\% &  12\% &   4\% &   9\% &  17\% &   4\% &   1\% &     58\% \\
			
			\hline \\[-1.8ex] 
			\multicolumn{1}{@{}l}{\textit{Non-POF Brokers}} \\ \\[-2.5ex] 
			\hline \\[-1.8ex] 
			
			Euro Pacific Capital   &      &  16\% &      &   3\% &   3\% &   3\% &  15\% &      &     59\% \\
			Elish Elish            &      &  30\% &      &      &      &  47\% &      &      &     23\% \\
			Florida Atlantic       &      &  26\% &      &   6\% &   2\% &   3\% &  12\% &      &     51\% \\
			LPL                    &      &  17\% &   7\% &  20\% &   6\% &   9\% &   7\% &      &     35\% \\
			Fifth Third            &      &  18\% &      &   2\% &   2\% &      &  13\% &      &     65\% \\
			Dakota Securities      &      &  \textless1\% &      &      &      &  41\% &      &      &     59\% \\
			BMO Capital            &      &      &  13\% &      &      &  \textless1\% &  \textless1\% &  \textless1\% &     87\% \\
			Bull Market Securities &      &  36\% &      &      &      &  29\% &      &      &     35\% \\
			Bank of the West       &  47\% &   7\% &      &  10\% &   4\% &   7\% &  10\% &      &     14\% \\
			AXA                    &      &  18\% &   8\% &  20\% &   7\% &   9\% &  12\% &      &     27\% \\
			Aurora Capital         &      &  44\% &      &      &      &  28\% &      &      &     28\% \\
			Edward Jones           &  \textless1\% &  31\% &   2\% &  21\% &   5\% &   7\% &   3\% &      &     32\% \\
			Benjamin Jerold        &      &  \textless1\% &      &      &      &  38\% &      &      &     62\% \\
			Insigneo Securities    &  56\% &   8\% &      &  10\% &   6\% &   7\% &  11\% &      &      2\% \\
			\textbf{Non-POF Average}      &  35\% &  19\% &   7\% &  11\% &   4\% &  18\% &   9\% &  \textless1\% &     41\% \\
			\hline\hline \\[-1.8ex] 
			\textbf{All Average} &  17\% &  17\% &  10\% &   9\% &   7\% &  17\% &   7\% &   1\% &     50\% \\
			\bottomrule
		\end{tabular}
		
	\end{table}
\end{center}


\begin{landscape}	

	\begin{table}[h]
		
		\caption*{\textbf{Table A2:} Summary Statistics for Execution Quality Variables}
		\centering
		\scriptsize
		
		\begin{tabular}{@{}lrrrrrrrr@{}}
			\multicolumn{9}{{@{}p{9.1in}@{}}}{
				{\scriptsize The following table was generated using 605 data from the fourth quarter of 2017 for market orders; each entry represents the average value over this time period for its respective variable. \textit{MktCtrExecShares} refers to the total number of shares executed at each market center as opposed to volume redirected to other market centers which is referred to as \textit{AwayExecShares}. \textit{PrImp\_AvgAmt} is the average amount (\$) of price improvement received by each share sent to a market center. \textit{PrImp\_Pct} refers to the percent of shares that received price improvement. The expected amount of price improvement, the product of average price improvement and the percent price improved, is given by \textit{PrImp\_ExpAmt}. \textit{All\_AvgT} refers to the average execution time (s) for a share sent to a market center, while \textit{PrImp\_AvgT} refers to the average execution time for shares that received price improvement. Lastly, \textit{AvgEffecSpread} is the average effective spread (\$) for each share executed; the effective spread of a trade is defined by the SEC as twice the difference between the trade price and the midpoint.} 
			}  \\ \\ [-1.8ex]
			\toprule
			\textbf{MarketCenter} &  \textbf{MktCtrExecShares} & \textbf{AwayExecShares} & \textbf{PrImp\_AvgAmt} & \textbf{PrImp\_Pct} &  \textbf{PrImp\_ExpAmt} &  \textbf{PrImp\_AvgT} &  \textbf{All\_AvgT} &  \textbf{AvgEffecSpread} \\
			\midrule
			\multicolumn{9}{@{}l}{\textit{Panel A: NASDAQ Listed Stocks}} \\ \\[-2.5ex] 
			\hline \\[-1.8ex] 
    		BNYC &          32693477 &               0 &      0.014408 &    86.80\% &      0.012507 &    0.142701 &  0.191631 &        0.020463 \\
			CDRG &         210788147 &          251506 &      0.024368 &    93.16\% &      0.022702 &    0.005730 &  0.005808 &        0.012677 \\
			FBCO &             70391 &         3874649 &      0.006053 &    84.15\% &      0.005093 &    0.004580 &  0.005972 &        0.059400 \\
			G1ES &          91879401 &               0 &      0.024455 &    94.90\% &      0.023208 &    0.007133 &  0.011400 &        0.015366 \\
			SGMA &          72691153 &               0 &      0.015809 &    87.26\% &      0.013795 &    0.001120 &  0.001922 &        0.022109 \\
			UBSS &          22122312 &        14882317 &      0.016574 &    94.33\% &      0.015635 &    0.017836 &  0.024544 &        0.019317 \\
			VRTU &         422915820 &               0 &      0.027917 &    89.02\% &      0.024851 &    0.049565 &  0.061613 &        0.019307 \\
			WOLV &           1023614 &               0 &      0.005317 &    90.86\% &      0.004831 &    0.004073 &  0.010679 &        0.029983 \\
			\hline \\[-1.8ex] 
			\multicolumn{9}{@{}l}{\textit{Panel B: NYSE Listed Stocks}} \\ \\[-2.5ex] 
			\hline \\[-1.8ex] 
			BNYC &          61096944 &               0 &      0.007877 &    90.81\% &      0.007153 &    0.139359 &  0.169840 &        0.009888 \\
			CDRG &         263908555 &          239049 &      0.012498 &    94.75\% &      0.011841 &    0.001010 &  0.001202 &        0.004549 \\
			FBCO &             73414 &         3233711 &      0.002994 &    86.23\% &      0.002582 &    0.011926 &  0.013332 &        0.031725 \\
			G1ES &         109132242 &               0 &      0.011672 &    96.53\% &      0.011266 &    0.004740 &  0.007217 &        0.007952 \\
			SGMA &          93677565 &               0 &      0.008363 &    88.43\% &      0.007395 &    0.002403 &  0.003950 &        0.012726 \\
			UBSS &          35612591 &        26536342 &      0.008811 &    95.05\% &      0.008375 &    0.004358 &  0.008671 &        0.010438 \\
			VRTU &         542823387 &               0 &      0.012047 &    89.59\% &      0.010793 &    0.173571 &  0.186829 &        0.010692 \\
			WOLV &            662979 &               0 &      0.003651 &    94.13\% &      0.003436 &    0.006349 &  0.007125 &        0.020637 \\
			\hline \\[-1.8ex] 
			\multicolumn{9}{@{}l}{\textit{Panel C: Other Exchange Listed Stocks}} \\ \\[-2.5ex] 
			\hline \\[-1.8ex]  
      	 	BNYC &          27178391 &               0 &      0.006149 &    86.39\% &      0.005312 &    0.157043 &  0.202053 &        0.009088 \\
			CDRG &         132598376 &          137748 &      0.011001 &    94.76\% &      0.010424 &    0.006075 &  0.008853 &        0.005234 \\
			FBCO &             11518 &         1238418 &      0.004315 &    86.51\% &      0.003733 &    0.047515 &  0.080076 &        0.032572 \\
			G1ES &          58899467 &               0 &      0.010127 &    97.04\% &      0.009828 &    0.012989 &  0.020845 &        0.006362 \\
			SGMA &          44349179 &               0 &      0.008467 &    92.29\% &      0.007814 &    0.001225 &  0.004434 &        0.008389 \\
			UBSS &          16530146 &        12972028 &      0.008205 &    96.84\% &      0.007946 &    0.024926 &  0.030768 &        0.009820 \\
			VRTU &         285378326 &               0 &      0.011482 &    90.48\% &      0.010390 &    0.111883 &  0.140918 &        0.011758 \\
			WOLV &            359806 &               0 &      0.002988 &    92.45\% &      0.002762 &    0.020072 &  0.024263 &        0.017517 \\
			\bottomrule
		\end{tabular}
		
	\end{table}

\end{landscape}


	
	








\pagebreak



%\section*{References}


\begin{thebibliography}{9}

	\bibitem{Angel} 
	Angel, J.J., Harris, L.E., \& Spatt, C. S. (2011). Equity Trading in the 21st Century. Quarterly Journal Of Finance, 1(1), 1-53.
	
	\bibitem{BCJ}
	Battalio, R., S. Corwin, S., \& Jennings, R.  (2016). Can Brokers Have It All? On the Relation between Make-Take Fees and Limit Order Execution Quality. Journal Of Finance, 71(5), 2193-2238. doi:10.1111/jofi.12422
	
	\bibitem{BSVN}
	Battalio, R., Shkilko, A., \& Van Ness, R. (2016). To Pay or Be Paid? The Impact of Taker Fees and Order Flow Inducements on Trading Costs in U.S. Options Markets. Journal Of Financial \& Quantitative Analysis, 51(5), 1637. doi:10.1017/S0022109016000582
	
	\bibitem{BH}
	Battalio, R., \& Holden, C. W. (2001). A simple model of payment for order flow, internalization, and total trading cost. Journal Of Financial Markets, 433-71. doi:10.1016/S1386-4181(00)00015-X
	
	\bibitem{chordia}
	Chordia, T., \& Subrahmanyam, A. (1995). Market Making, the Tick Size, and Payment-for-Order Flow: Theory and Evidence. The Journal Of Business, (4), 543.
	
	\bibitem{Cimon}
	Cimon, D. (2016). Broker routing decisions in limit order markets. Bank of Canada. \href{
		http://www.bankofcanada.ca/wp-content/uploads/2016/11/swp2016-50.pdf}{\textit{
			http://www.bankofcanada.ca/wp-content/uploads/2016/11/swp2016-50.pdf}}
	
	\bibitem{dutta}
	Dutta, P.K., \& Madhavan, A. (1997). Competition and Collusion in Dealer Markets. The Journal Of Finance, (1), 245. doi:10.2307/2329563
	
	
	\bibitem{kandel}
	Kandel, E., \&  Marx, L.M. (1999). Payments for Order Flow on Nasdaq. The Journal Of Finance, (1), 35.
	
	\bibitem{NASD}
	Financial Industry Regulatory Authority, 2001. NASD Notice to Members 01-22. \href{http://www.finra.org/industry/notices/01-22}{http://www.finra.org/industry/notices/01-22}
	
	\bibitem{jurgen}
	Dennert, J. (1993). Price Competition between Market Makers. The Review Of Economic Studies, (3), 735.
	
	\bibitem{ohara} 
	O’Hara, M. \& Ye, M. (2011) Is market fragmentation harming market quality? Journal of
	Financial Economics 100, 459-474
	
	\bibitem{Maglaras}
	Maglaras, C., Moallemi, C., \& Zheng, H. (2015). Optimal Execution in a Limit Order Book and an Associated Microstructure Market Impact Model. Columbia Business School Research Paper No. 15-60. 
	
	\bibitem{parlour}
	Parlour, C.A., \& Rajan, U. (2003). Payment for order flow. Journal Of Financial Economics, 68379-411. doi:10.1016/S0304-405X(03)00071-0
	
	\bibitem{NMS}
	Securities and Exchange Commision, 2005. SEC adopts regulation NMS and provisions regarding Investment Advisers Act of 1940. \href{https://www.sec.gov/news/press/2005-48.htm}{\textit{https://www.sec.gov/news/press/2005-48.htm}}
	
\end{thebibliography}	

\end{document}





